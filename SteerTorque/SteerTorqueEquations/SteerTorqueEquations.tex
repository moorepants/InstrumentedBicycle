\documentclass[]{article}
\usepackage{amssymb,amsmath}
\newcommand{\sgn}{\operatorname{sgn}}

\title{On Calculating the True Bicycle Steer Torque}
\author{Jason K. Moore}
\date{July 29, 2011}

\begin{document}
\maketitle
We measure the torque in the steering column, $T_M$, from a sensor that is
mounted between both the handlebars and fork steer tube and two sets of
bearings: the headset and the slip clutch bearings. We are interested in
knowing the torque applied about the steer axis by the rider's contact forces
to the handlebars, $T_\delta$. It turns out that this is a function of much of
the data measured on the bicycle.

A free body diagram can be drawn of the upper portion of the handlebar/fork
assembly, where the lower portion is cut at the steer torque sensor. The
torques acting on the handlebar about the steer axis are the measured torque,
$T_M$, the rider applied steer torque, $T_\delta$, and the friction from the
upper bearing set which can be described by coulomb, $T_F$, and viscous
friction, $T_V$.

The coulomb friction can be described as:
%
\begin{equation}
	T_F = F\sgn(\dot\delta) = \left\{
		\begin{array}{rl}
			F & \textrm{if $\dot{\delta}>0$}\\
			0 & \textrm{if $\dot{\delta}=0$}\\
			-F & \textrm{if $\dot{\delta}<0$}\\
		\end{array}
	\right.
\end{equation}
%
and viscous friction as:
%
\begin{equation}
	T_V = c\dot{\delta}
\end{equation}
%
where $F$ is the coulomb friction force and $c$ is the viscous damping
coefficient.

We measure the angular rate of the bicycle frame, $B$, with three rate gyros:
%
\begin{equation}
	^N\bar{\omega}^B = w_{b1}\hat{b}_1 + w_{b2}\hat{b}_2 + w_{b3}\hat{b}_3
\end{equation}
%
The handlebar, $H$, is connected to the bicycle frame, $B$, by a revolute joint
that rotates through the steering angle, $\delta$, and we measure the angular
rate of the handlebar about the steer axis directly with a rate gyro. The
angular rate of the handlebar can be written as follows:
%
\begin{equation}
	^N\bar{\omega}^H = (w_{b1}\cos(\delta) + w_{b2}\sin(\delta))\hat{h}_1 +
		(-w_{b1}\sin(\delta) + w_{b2}\cos(\delta))\hat{h}_2 + w_{h3}\hat{h}_3
\end{equation}
%
The steer rate, $\dot{\delta}$, can be computed by subtracting the angular rate
of the bicycle frame about the steer axis from the angular rate of the
handlebar/fork about the steer axis.
%
\begin{equation}
	\dot{\delta} = w_{h3} - w_{b3}
\end{equation}
%
We measure the acceleration of a point, $v$, on the bicycle frame.
%
\begin{equation}
	^N\bar{a}^v = a_{v1}\hat{b}_1 + a_{v2}\hat{b}_2 + a_{v3}\hat{b}_3
\end{equation}
%
We also know the location of a point on the steer axis, $s$, relative to point
$v$.
%
\begin{equation}
	\bar{r}^{s/v} = d_{s1}\hat{b}_1 + d_{s3}\hat{b}_3
\end{equation}
%
The location of the center of mass of the handlebar, $h_o$, is also known
relative to point $s$.
%
\begin{equation}
	\bar{r}^{h_o/s} = d\hat{h}_1
\end{equation}
%
$^N\bar{a}^{h_o}$ can be calculated using the two point thereom for
acceleration twice staring at the point $v$:
%
\begin{equation}
	^N\bar{a}^s = ^N\bar{a}^v + ^N\dot{\bar{\omega}}^B\times\bar{r}^{s/v} +
		^N\bar{\omega}^B\times(^N\bar{\omega}^B\times\bar{r}^{s/v})
\end{equation}
%
\begin{equation}
	^N\bar{a}^{h_o} = ^N\bar{a}^s + ^N\dot{\bar{\omega}}^H\times\bar{r}^{h_o/s} +
		^N\bar{\omega}^H\times(^N\bar{\omega}^H\times\bar{r}^{h_o/s})
\end{equation}
%
The angular momentum of the handlebar about its center of mass is:
%
\begin{equation}
	^N\bar{H}^{H/h_o} = I^{H/h_o} \cdot ^N\bar{\omega}^H
\end{equation}
%
where $I^{H/h_o}$ is the inertia dyadic with reference to the center of mass
which exhibits symmetry about the $13$-plane.

The dynamic equations of motion of the handlebar can be written as the sum of
the torques on the handlebar about point $s$ is equal to the derivative of the
angular momentum of $H$ in $N$ about $h_o$ plus the cross product of the vector
from $s$ to $h_o$ with the mass times the acceleration of $h_o$ in $N$:
%
\begin{equation}
	\sum \bar{T}^{H/s} = ^N\dot{\bar{H}}^{H/h_o} + \bar{r}^{h_o/s} \times m_H
		\ ^N\bar{a}^{ho}
\end{equation}
%
The only torques applied to the handlebar that we are interested in act about the steer axis.
\begin{equation}
	\sum T^{H/s}_3 = T_\delta - T_F - T_M - T_V
\end{equation}
%
Looking at only the 3 component of the equation of motion gives the following
relationship:
%
\begin{equation}
	T_\delta - F\sgn({\dot{\delta}}) - T_M - c(w_{h3} - w_{b3}) = (^N\dot{\bar{H}}^{H/h_o} + ^s\bar{r}^h_o \times m_H
		\ ^N\bar{a}^s) \cdot \hat{h}_3
\end{equation}
%
And $T_\delta$ can be solved for:
\begin{align}
	T_{\delta} = &
	I_{H33} \dot{w}_{h3} + \\\notag
	%
	& (I_{H11} (w_{b1}\cos(\delta) +
	w_{b2}\sin(\delta)) +
	I_{H31} w_{h3}) (-w_{b1}\sin(\delta) +
	w_{b2}\cos(\delta)) + \\\notag
	%
	& I_{H22} (- w_{b1} \sin(\delta) +
	w_{b2}\cos(\delta))
	(w_{b1}\cos(\delta) +
	w_{b2}\sin(\delta)) + \\\notag
	%
	& I_{H31} (- (- w_{b3} + w_{h3}) w_{b1}
	\sin(\delta) +
	(-w_{b3} + w_{h3})
	w_{b2}\cos(\delta) +
	\sin(\delta)\dot{w}_{b2} +
	\cos(\delta)\dot{w}_{b1}) +\\\notag
	%
	& d m_H (d (-w_{b1}\sin(\delta) + w_{b2}
	\cos(\delta))(w_{b1}\cos(\delta) +
	w_{b2}\sin(\delta)) +
	d \dot{w}_{h3}) - \\\notag
	%
	& d m_H (- d_{s1} w_{b2}^{2} + d_{s2}
	\dot{w}_{b2} - (d_{s1}
	w_{b3} - d_{s2}
	w_{b1}) w_{b3} +
	a_{v1})
	\sin(\delta) +\\\notag
	%
	& d m_H(d_{s1} w_{b1}w_{b2} +
	d_{s1} \dot{w}_{b3} +
	d_{s2} w_{b2} w_{b3} - d_{s2} \dot{w}_{b1} +
	a_{v2})\cos(\delta) - \\\notag
%
	& c (- w_{b3} + w_{h3}) + T_F + T_s
\end{align}
\end{document}
